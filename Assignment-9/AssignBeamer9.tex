\documentclass{beamer}
\usepackage{listings}
\usepackage{blkarray}
\usepackage{listings}
\usepackage{subcaption}
\usepackage{url}
\usepackage{tikz}
\usepackage{tkz-euclide} % loads  TikZ and tkz-base
%\usetkzobj{all}
\usetikzlibrary{calc,math}
\usepackage{float}
\newcommand\norm[1]{\left\lVert#1\right\rVert}
\renewcommand{\vec}[1]{\mathbf{#1}}
\usepackage[export]{adjustbox}
\usepackage[utf8]{inputenc}
\usepackage{amsmath}
\usepackage{amsfonts}
\usepackage{tikz}
\usepackage{hyperref}
\usepackage{multirow}
\usepackage{bm}
\hypersetup{
    colorlinks = true,
    linkbordercolor = {white},
    linkcolor={red},
    citecolor={green},
    filecolor={blue},
	menucolor={red},
	runcolor={cyan},
	urlcolor={blue},
	breaklinks=true
}
\usetikzlibrary{automata, positioning}
\usetheme{Boadilla}
\providecommand{\pr}[1]{\ensuremath{\Pr\left(#1\right)}}
\providecommand{\mbf}{\mathbf}
\providecommand{\qfunc}[1]{\ensuremath{Q\left(#1\right)}}
\providecommand{\sbrak}[1]{\ensuremath{{}\left[#1\right]}}
\providecommand{\lsbrak}[1]{\ensuremath{{}\left[#1\right.}}
\providecommand{\rsbrak}[1]{\ensuremath{{}\left.#1\right]}}
\providecommand{\brak}[1]{\ensuremath{\left(#1\right)}}
\providecommand{\lbrak}[1]{\ensuremath{\left(#1\right.}}
\providecommand{\rbrak}[1]{\ensuremath{\left.#1\right)}}
\providecommand{\cbrak}[1]{\ensuremath{\left\{#1\right\}}}
\providecommand{\lcbrak}[1]{\ensuremath{\left\{#1\right.}}
\providecommand{\rcbrak}[1]{\ensuremath{\left.#1\right\}}}
\providecommand{\abs}[1]{\vert#1\vert}
\newcommand*{\permcomb}[4][0mu]{{{}^{#3}\mkern#1#2_{#4}}}
\newcommand*{\perm}[1][-3mu]{\permcomb[#1]{P}}
\newcommand*{\comb}[1][-1mu]{\permcomb[#1]{C}}

\newcounter{saveenumi}
\newcommand{\seti}{\setcounter{saveenumi}{\value{enumi}}}
\newcommand{\conti}{\setcounter{enumi}{\value{saveenumi}}}

\makeatletter
\newenvironment<>{proofs}[1][\proofname]{%
    \par
    \def\insertproofname{#1\@addpunct{.}}%
    \usebeamertemplate{proof begin}#2}
  {\usebeamertemplate{proof end}}
\makeatother
%% Theme choice:
%\usetheme{CambridgeUS}

% Title page details: 
\title{\textbf{Assignment 9\\ \Large AI1110: Probability and Random Variables \\ \large Indian Institute of Technology, Hyderabad}}
\author{JARUPULA SAI KUMAR\\ CS21BTECH11023}

\date{\today}

\begin{document}
	% The title page
	\begin{frame}
		\titlepage
	\end{frame}
	
	% The table of contents
	\begin{frame}{Outline}
    		\tableofcontents
	\end{frame}
	
	% The question
	\section{Question}
	\begin{frame}{Question 10.10}
       If $R_n(\tau) = N\delta(\tau)$ and\\
       $x(t) = A\cos\omega_0t + n(t)$\\
       $H(\omega) = \dfrac{1}{\alpha+j\omega}$\\
       $y(t) = B\cos(\omega_0 +t +\phi) + y_n(t)$\\
       where $y_n(t)$ is the component of the output y(t) due to n(t), find the value of $\alpha$ that maximises the signal to noise ratio $\dfrac{B^2}{E(y_n^2(t))}$
	\end{frame}
	
	% The solution
	\section{Solution}
	\begin{frame}{Solution}
	    We Know that,\\
	    \begin{align}
	  B &= A|H(\omega_0)| = \dfrac{A}{\sqrt{\alpha^2+\omega_0^2}}\\
	    S_{y_n}(\omega) &= \dfrac{N}{\alpha^2+\omega_0^2}\\
	    R_{y_n}(\tau) &= \dfrac{N}{2\alpha e^{-\alpha|\tau|}}\\
	    E{y_n^2(t)} &= R_{y_n}(0) = \dfrac{N}{2\alpha}
	    \end{align}
	 \end{frame}
	 
	 % The final answer
	\section{Answer}
	\begin{frame}{Answer}
	   Hence,We can written it as,\\
	         $\dfrac{B^2}{E(y_n^2(t))} = \dfrac{2A^2}{N}\dfrac{\alpha}{\alpha^2+\omega_0^2}$
	Differentiating, we get
	\begin{align}
	\dfrac{1(\alpha^2+\omega_0^2)-\alpha(2\alpha)}{(\alpha^2+\omega_0^2)^2} &= 0\\
	\omega_0^2 - \alpha^2 &= 0\\
	\alpha &= \omega_0 
	\end{align}
	Also,$f^{\prime\prime}(\alpha)<0$\\
	$\therefore\alpha = \omega_0$ which is the maxima value for given             ratio.
	\end{frame}
\end{document}