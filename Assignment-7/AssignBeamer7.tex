\let\negmedspace\undefined
\let\negthickspace\undefined
\documentclass{beamer}
\usetheme{CambridgeUS}
\usepackage{csquotes}
\usepackage{comment}
\usepackage{enumerate}
\usepackage{amsmath,amssymb,amsthm}
\usepackage{graphicx}
\let\vec\mathbf
\newcommand{\myvec}[1]{\ensuremath{\begin{pmatrix}#1\end{pmatrix}}}
\providecommand{\brak}[1]{\ensuremath{\left(#1\right)}}
\providecommand{\pr}[1]{\ensuremath{\Pr\left(#1\right)}}


     \title{Assignment 7}
     \author{JARUPULA SAI KUMAR \\
     CS21BTECH11023 }
     \date{\today}
\logo{\large \LaTeX{}}
     
     \begin{document}
     \begin{frame}
     \maketitle    
     \end{frame}
     
     \logo{}
     
     \begin{frame}{Outline}
    \tableofcontents
     \end{frame}
      \section{Question}
      \begin{frame}
          

      \textbf{Question:}The random variable $X$ has the Erlang density $f(x)~c^{4}.x^{3}.e^{-cx}$. We observe the samples $X_{i} = 3.1, 3.4.3.3$ Find the ML estimate c.\\
           \end{frame}
           
         \section{Solution:}
         \begin{frame}
             \textbf{Solution:}Lets generalize and find ML.Lets take for $n$
     random values the p.d.f will be
     \begin{align}
         &f(x,c)=c^{4}.x^{3}.e^{-cx}\\
         &f(x_{1},x_{2},x_{3},..x_{n},c)=c^{4n}.(x_{1}..x_{n})^{3}.e^{-nc\hat{x}}
         \end{align}
         \end{frame}
         \begin{frame}
             Where $\hat{x}$ is the mean of random variable $X$.\\
         Now to find ML of this function partially differentiate it w.r.t $c$.
         \begin{align}
             &\frac{\partial f(X,c)}{\partial c}=\frac{\partial c^{4n}.(x_{1}..x_{n})^{3}.e^{-n^c\hat{x}}}{\partial c}\\
             &\frac{\partial f(X,c)}{\partial c}=n.c^{4n-1}.x^{3}e^{-cn\hat{x}}(4-c\hat{x})
         \end{align}
         \end{frame}
         \begin{frame}
             for ML equate partial differentiation to zero and that value is the estimate of c,
         \begin{align}
             n.c^{4n-1}.x^{3}e^{-cn\hat{x}}(4-c\hat{x})=0\\
             4-c\hat{x}=0\\
             c=\frac{4}{\hat{x}}
         \end{align}
         
         \end{frame}
         \begin{frame}
          given $X_{i} = 3.1, 3.4.3.3,$
         \begin{align}
        \hat{x}&=\frac{3.1+3.3+3.4}{3}\\
              \hat{x}&=3.27
         \end{align}
             Now,
         \begin{align}
         c&=\frac{4}{\hat{x}}\\
         c&=\frac{4}{3.27}\\
         c&=1.224.
         \end{align}
         \end{frame}
     \end{document}