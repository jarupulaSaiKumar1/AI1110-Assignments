\documentclass{beamer}
\usepackage{listings}
\usepackage{blkarray}
\usepackage{listings}
\usepackage{subcaption}
\usepackage{url}
\usepackage{tikz}
\usepackage{tkz-euclide} % loads  TikZ and tkz-base
%\usetkzobj{all}
\usetikzlibrary{calc,math}
\usepackage{float}
\newcommand\norm[1]{\left\lVert#1\right\rVert}
\renewcommand{\vec}[1]{\mathbf{#1}}
\usepackage[export]{adjustbox}
\usepackage[utf8]{inputenc}
\usepackage{amsmath}
\usepackage{amsfonts}
\usepackage{tikz}
\usepackage{hyperref}
\usepackage{multirow}
\usepackage{bm}
\hypersetup{
    colorlinks = true,
    linkbordercolor = {white},
    linkcolor={red},
    citecolor={green},
    filecolor={blue},
	menucolor={red},
	runcolor={cyan},
	urlcolor={blue},
	breaklinks=true
}
\usetikzlibrary{automata, positioning}
\usetheme{Boadilla}
\providecommand{\pr}[1]{\ensuremath{\Pr\left(#1\right)}}
\providecommand{\mbf}{\mathbf}
\providecommand{\qfunc}[1]{\ensuremath{Q\left(#1\right)}}
\providecommand{\sbrak}[1]{\ensuremath{{}\left[#1\right]}}
\providecommand{\lsbrak}[1]{\ensuremath{{}\left[#1\right.}}
\providecommand{\rsbrak}[1]{\ensuremath{{}\left.#1\right]}}
\providecommand{\brak}[1]{\ensuremath{\left(#1\right)}}
\providecommand{\lbrak}[1]{\ensuremath{\left(#1\right.}}
\providecommand{\rbrak}[1]{\ensuremath{\left.#1\right)}}
\providecommand{\cbrak}[1]{\ensuremath{\left\{#1\right\}}}
\providecommand{\lcbrak}[1]{\ensuremath{\left\{#1\right.}}
\providecommand{\rcbrak}[1]{\ensuremath{\left.#1\right\}}}
\providecommand{\abs}[1]{\vert#1\vert}
\newcommand*{\permcomb}[4][0mu]{{{}^{#3}\mkern#1#2_{#4}}}
\newcommand*{\perm}[1][-3mu]{\permcomb[#1]{P}}
\newcommand*{\comb}[1][-1mu]{\permcomb[#1]{C}}

\newcounter{saveenumi}
\newcommand{\seti}{\setcounter{saveenumi}{\value{enumi}}}
\newcommand{\conti}{\setcounter{enumi}{\value{saveenumi}}}

\makeatletter
\newenvironment<>{proofs}[1][\proofname]{%
    \par
    \def\insertproofname{#1\@addpunct{.}}%
    \usebeamertemplate{proof begin}#2}
  {\usebeamertemplate{proof end}}
\makeatother
%% Theme choice:
%\usetheme{CambridgeUS}

% Title page details: 
\title{\textbf{Assignment 8\\ \Large AI1110: Probability and Random Variables \\ \large Indian Institute of Technology, Hyderabad}}
\author{JARUPULA SAI KUMAR\\ CS21BTECH11023}

\date{\today}
\logo{\large \LaTeX{}}
\begin{document}
\newcommand{\BEQA}{\begin{eqnarray}}
\newcommand{\EEQA}{\end{eqnarray}}
\newcommand{\define}{\stackrel{\triangle}{=}}
\newcommand*\circled[1]{\tikz[baseline=(char.base)]{
    \node[shape=circle,draw,inner sep=2pt] (char) {#1};}}
\bibliographystyle{IEEEtran}
	\begin{frame}
		\titlepage
	\end{frame}
\begin{frame}{Outline}
\tableofcontents
\end{frame}
\begin{frame}{Abstract}
     \begin{block}{Abstract}
       This Presentation contains the detailed solution of  problem 10.9 of chapter 10 in the famous  papoullis textbook.   
      \end{block}
\end{frame}

\end{frame}
\section{Problem}
\begin{frame}{Problem}
\begin{block}{Chapter 10-10.9}
The position of a particle in underdamped harmonic motion is a normal process with
autocorrelation as in (10-60).  Show that its conditional density assuming $x(O) = X_o$ and
$x'(O) = v(O) = V_o$ equals
\begin{align}
f\brak{x|x_o , v_o} = \frac{1}{\sqrt{2\pi P}}e^{-\frac{\brak{x-ax_o-bv_o}^2}{2}}
\end{align}
Find the  constants a, b, and P. 
\end{block}
\begin{block}{}
\begin{align}
R_{x}(\tau)=\frac{k T}{c} e^{-\alpha|\tau|}\left(\cos \beta \tau+\frac{\alpha}{\beta} \sin \beta|\tau|\right)
\end{align}
\end{block}
\end{frame}
\section{Solution}
\begin{frame}{Solution}

Let us try to use the given conditions to solve the constants a,b and P
	\begin{align}
	R_{x}(\tau)=\frac{k T}{c} e^{-\alpha|\tau|}\left(\cos \beta \tau+\frac{\alpha}{\beta} \sin \beta|\tau|\right) \\
	\underset{\sim}{x}(\tau)-a \underset{\sim}{x}(0)-b \underset{\sim}{v}(0) \perp \underset{\sim}{x}(0), \underset{\sim}{v}(0) \\
	\end{align}
	We know that,
	\begin{align}
&R_{x x}(\tau)=a R_{x x}(0)+b R_{x v}(0) \\
&R_{x v}(\tau)=a R_{x v}(0)+b R_{v v}(0)
\end{align}
\end{frame}
\begin{frame}
\begin{align}
&R_{x x}(\tau)=A e^{-\alpha \tau}\left(\cos B \tau+\frac{\alpha}{B} \sin B \tau\right) \\
&R_{x v}(\tau)=-R_{x x}^{\prime}(\tau)=A e^{-\alpha \tau}(\sin B \tau) \frac{\alpha^{2}+3^{2}}{\beta} \\
&R_{v v}(\tau)=R_{x v}^{\prime}(\tau)=A e^{-\alpha \tau}\left(\cos B \tau-\frac{\alpha}{B} \sin B \tau\right) \frac{\alpha^{2}+\beta^{2}}{\beta^{2}} \\
&\text { Inserting into (i) and solving, we obtain } \\
&a=e^{-\alpha \tau}\left(\cos B \tau+\frac{\alpha}{B} \sin B \tau\right) \\
&b=\frac{1}{B} e^{-\alpha \tau} \sin B \tau
\end{align}
\end{frame}
\section{Final Result}
\begin{frame}{final Result}
$$
\begin{aligned}
P &=E\left\{\left[x(t)-a x(0)-bv(0)\right] x(t)\right\}=R_{x x}(0)-a R_{x x}(t)-b R_{x v}(t) \\
&=\frac{2 k T f}{m^{2}}\left[1-e^{-2 \alpha t}\left(1+\frac{2 \alpha^{2}}{B} \sin ^{2} B t+\frac{\alpha}{B} \sin 2 \beta t\right)\right]
\end{aligned}
$$
\end{frame}
\end{document}