\documentclass[journal,12pt,twocolumn]{IEEEtran}
%
\usepackage{setspace}
\usepackage{gensymb}
%\doublespacing
\singlespacing

%\usepackage{graphicx}
%\usepackage{amssymb}
%\usepackage{relsize}
\usepackage[cmex10]{amsmath}
%\usepackage{amsthm}
%\interdisplaylinepenalty=2500
%\savesymbol{iint}
%\usepackage{txfonts}
%\restoresymbol{TXF}{iint}
%\usepackage{wasysym}
\usepackage{amsthm}
%\usepackage{iithtlc}
\usepackage{mathrsfs}
\usepackage{txfonts}
\usepackage{stfloats}
\usepackage{bm}
\usepackage{cite}
\usepackage{cases}
\usepackage{subfig}
%\usepackage{xtab}
\usepackage{longtable}
\usepackage{multirow}
%\usepackage{algorithm}
%\usepackage{algpseudocode}
\usepackage{booktabs}
\usepackage{enumitem}
\usepackage{mathtools}
\usepackage{tikz}
\usepackage{pgfplots}
\usepackage{circuitikz}
\usepackage{verbatim}
\usepackage{tfrupee}
\usepackage[breaklinks=true]{hyperref}
%\usepackage{stmaryrd}
\usepackage{tkz-euclide} % loads  TikZ and tkz-base
%\usetkzobj{all}
\usetikzlibrary{fit}
\usetikzlibrary{calc,math}
%\pgfdeclarelayer{background}
%\pgfsetlayers{background}
\usepackage{listings}
    \usepackage{color}                                            %%
    \usepackage{array}                                            %%
    \usepackage{longtable}                                        %%
    \usepackage{calc}                                             %%
    \usepackage{multirow}                                         %%
    \usepackage{hhline}                                           %%
    \usepackage{ifthen}                                           %%
  %optionally (for landscape tables embedded in another document): %%
    \usepackage{lscape}     
\usepackage{multicol}
\usepackage{chngcntr}
%\usepackage{enumerate}

%\usepackage{wasysym}
%\newcounter{MYtempeqncnt}
\DeclareMathOperator*{\Res}{Res}
%\renewcommand{\baselinestretch}{2}
\renewcommand\thesection{\arabic{section}}
\renewcommand\thesubsection{\thesection.\arabic{subsection}}
\renewcommand\thesubsubsection{\thesubsection.\arabic{subsubsection}}

\renewcommand\thesectiondis{\arabic{section}}
\renewcommand\thesubsectiondis{\thesectiondis.\arabic{subsection}}
\renewcommand\thesubsubsectiondis{\thesubsectiondis.\arabic{subsubsection}}

% correct bad hyphenation here
\hyphenation{op-tical net-works semi-conduc-tor}
\def\inputGnumericTable{}                                 %%

\lstset{
%language=C,
frame=single, 
breaklines=true,
columns=fullflexible
}
%\lstset{
%language=tex,
%frame=single, 
%breaklines=true
%}

\begin{document}
%


\newtheorem{theorem}{Theorem}[section]
\newtheorem{problem}{Problem}
\newtheorem{proposition}{Proposition}[section]
\newtheorem{lemma}{Lemma}[section]
\newtheorem{corollary}[theorem]{Corollary}
\newtheorem{example}{Example}[section]
\newtheorem{definition}[problem]{Definition}
%\newtheorem{thm}{Theorem}[section] 
%\newtheorem{defn}[thm]{Definition}
%\newtheorem{algorithm}{Algorithm}[section]
%\newtheorem{cor}{Corollary}
\newcommand{\BEQA}{\begin{eqnarray}}
\newcommand{\EEQA}{\end{eqnarray}}
\newcommand{\define}{\stackrel{\triangle}{=}}
\newcommand{\RomanNumeralCaps}[1]
    {\MakeUppercase{\romannumeral #1}}
\bibliographystyle{IEEEtran}
%\bibliographystyle{ieeetr}
\providecommand{\mbf}{\mathbf}
\providecommand{\pr}[1]{\ensuremath{\Pr\left(#1\right)}}
\providecommand{\qfunc}[1]{\ensuremath{Q\left(#1\right)}}
\providecommand{\sbrak}[1]{\ensuremath{{}\left[#1\right]}}
\providecommand{\lsbrak}[1]{\ensuremath{{}\left[#1\right.}}
\providecommand{\rsbrak}[1]{\ensuremath{{}\left.#1\right]}}
\providecommand{\brak}[1]{\ensuremath{\left(#1\right)}}
\providecommand{\lbrak}[1]{\ensuremath{\left(#1\right.}}
\providecommand{\rbrak}[1]{\ensuremath{\left.#1\right)}}
\providecommand{\cbrak}[1]{\ensuremath{\left\{#1\right\}}}
\providecommand{\lcbrak}[1]{\ensuremath{\left\{#1\right.}}
\providecommand{\rcbrak}[1]{\ensuremath{\left.#1\right\}}}
\theoremstyle{remark}
\newtheorem{rem}{Remark}
\newcommand{\sgn}{\mathop{\mathrm{sgn}}}
\providecommand{\abs}[1]{\left\vert#1\right\vert}
\providecommand{\res}[1]{\Res\displaylimits_{#1}} 
\providecommand{\norm}[1]{\left\lVert#1\right\rVert}
%\providecommand{\norm}[1]{\lVert#1\rVert}
\providecommand{\mtx}[1]{\mathbf{#1}}
\providecommand{\mean}[1]{E\left[ #1 \right]}
\providecommand{\fourier}{\overset{\mathcal{F}}{ \rightleftharpoons}}
%\providecommand{\hilbert}{\overset{\mathcal{H}}{ \rightleftharpoons}}
\providecommand{\system}{\overset{\mathcal{H}}{ \longleftrightarrow}}
	%\newcommand{\solution}[2]{\textbf{Solution:}{#1}}
\newcommand{\solution}{\noindent \textbf{Solution: }}
\newcommand{\cosec}{\,\text{cosec}\,}
\providecommand{\dec}[2]{\ensuremath{\overset{#1}{\underset{#2}{\gtrless}}}}
\newcommand{\myvec}[1]{\ensuremath{\begin{pmatrix}#1\end{pmatrix}}}
\newcommand{\mydet}[1]{\ensuremath{\begin{vmatrix}#1\end{vmatrix}}}
\newcommand*{\permcomb}[4][0mu]{{{}^{#3}\mkern#1#2_{#4}}}
\newcommand*{\perm}[1][-3mu]{\permcomb[#1]{P}}
\newcommand*{\comb}[1][-1mu]{\permcomb[#1]{C}}
%\newcommand*{\perm}[2]{{}^{#1}\!P_{#2}}%
%\newcommand*{\comb}[2]{{}^{#1}C_{#2}}%
%\numberwithin{equation}{section}
%\numberwithin{equation}{subsection}
%\numberwithin{problem}{section}
%\numberwithin{definition}{section}
\makeatletter
\@addtoreset{figure}{problem}
\makeatother
\let\StandardTheFigure\thefigure
\let\vec\mathbf
%\renewcommand{\thefigure}{\theproblem.\arabic{figure}}
\renewcommand{\thefigure}{\theproblem}
%\setlist[enumerate,1]{before=\renewcommand\theequation{\theenumi.\arabic{equation}}
%\counterwithin{equation}{enumi}
%\renewcommand{\theequation}{\arabic{subsection}.\arabic{equation}}
\def\putbox#1#2#3{\makebox[0in][l]{\makebox[#1][l]{}\raisebox{\baselineskip}[0in][0in]{\raisebox{#2}[0in][0in]{#3}}}}
     \def\rightbox#1{\makebox[0in][r]{#1}}
     \def\centbox#1{\makebox[0in]{#1}}
     \def\topbox#1{\raisebox{-\baselineskip}[0in][0in]{#1}}
     \def\midbox#1{\raisebox{-0.5\baselineskip}[0in][0in]{#1}}
     
\title{\textbf{Assignment 8\\ \Large AI1110: Probability and Random Variables \\ \large Indian Institute of Technology, Hyderabad}}
\author{JARUPULA SAI KUMAR\\ CS21BTECH11023}
\maketitle{}
\textbf{Question 10.9 [ Papoulis Textbook ]:}
The position of a particle in underdamped harmonic motion is a normal process with
autocorrelation as in (10-60).  Show that its conditional density assuming $x(O) = X_o$ and
$x'(O) = v(O) = V_o$ equals
\begin{align}
f\brak{x|x_o , v_o} = \frac{1}{\sqrt{2\pi P}}e^{-\frac{\brak{x-ax_o-bv_o}^2}{2}}
\end{align}
Find the  constants a, b, and P. 
\begin{align}
R_{x}(\tau)=\frac{k T}{c} e^{-\alpha|\tau|}\left(\cos \beta \tau+\frac{\alpha}{\beta} \sin \beta|\tau|\right)
\end{align}

\section{Solution}
\begin{frame}

Let us try to use the given conditions to solve the constants a,b and P
	\begin{align}
	R_{x}(\tau)=\frac{k T}{c} e^{-\alpha|\tau|}\left(\cos \beta \tau+\frac{\alpha}{\beta} \sin \beta|\tau|\right) \\
	\underset{\sim}{x}(\tau)-a \underset{\sim}{x}(0)-b \underset{\sim}{v}(0) \perp \underset{\sim}{x}(0), \underset{\sim}{v}(0) \\
	\end{align}
	We know that,
	\begin{align}
&R_{x x}(\tau)=a R_{x x}(0)+b R_{x v}(0) \\
&R_{x v}(\tau)=a R_{x v}(0)+b R_{v v}(0)
\end{align}
\end{frame}
\begin{frame}

&R_{x x}(\tau)=A e^{-\alpha \tau}\left(\cos B \tau+\frac{\alpha}{B} \sin B \tau\right) \\
&R_{x v}(\tau)=-R_{x x}^{\prime}(\tau)=A e^{-\alpha \tau}(\sin B \tau) \frac{\alpha^{2}+3^{2}}{\beta} \\
&R_{v v}(\tau)=R_{x v}^{\prime}(\tau)=A e^{-\alpha \tau}\left(\cos B \tau-\frac{\alpha}{B} \sin B \tau\right) \frac{\alpha^{2}+\beta^{2}}{\beta^{2}} 

&\text { Inserting into (i) and solving, we obtain } \\
&a=e^{-\alpha \tau}\left(\cos B \tau+\frac{\alpha}{B} \sin B \tau\right) \\
&b=\frac{1}{B} e^{-\alpha \tau} \sin B \tau
\end{align}
\end{frame}
\begin{aligned}
P &=E\left\{\left[x(t)-a x(0)-bv(0)\right] x(t)\right\}=R_{x x}(0)-a R_{x x}(t)-b R_{x v}(t) \\
&=\frac{2 k T f}{m^{2}}\left[1-e^{-2 \alpha t}\left(1+\frac{2 \alpha^{2}}{B} \sin ^{2} B t+\frac{\alpha}{B} \sin 2 \beta t\right)\right]
\end{aligned}
$$
\end{enumerate}
\end{document}