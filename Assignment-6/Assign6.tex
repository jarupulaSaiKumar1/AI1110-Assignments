\let\negmedspace\undefined
\let\negthickspace\undefined
\documentclass[journal,12pt,twocolumn]{IEEEtran}

\usepackage{csquotes}
\usepackage{comment}
\usepackage{enumerate}
\usepackage{amsmath,amssymb,amsthm}
\usepackage{graphicx}
\let\vec\mathbf
\newcommand{\myvec}[1]{\ensuremath{\begin{pmatrix}#1\end{pmatrix}}}
\providecommand{\brak}[1]{\ensuremath{\left(#1\right)}}
\providecommand{\pr}[1]{\ensuremath{\Pr\left(#1\right)}}
 \title{\textbf{Assignment 6\\ \Large AI1110: Probability and Random Variables \\ \large Indian Institute of Technology, Hyderabad}}
     \author{JARUPULA SAI KUMAR \\
     CS21BTECH11023\\}
     
     \begin{document}
     \maketitle
     \textbf{Question:}The readings of a voltmeter introduces an error $nu$ with mean $0$. We wish to estimate its standard deviation $\sigma$. We measure a calibrated source V = 3 V four times and obtain the values $ 2.90, 3.15 , 3.05, 2.96 $ Assuming that $\nu$ is normal, find the 0.95 confidence interval of $\sigma$.
      \end{block}
      
\end{frame}\\
\begin{block}{Solution}
        So,We are having 4 observations like  $ 2.90, 3.15 , 3.05, 2.96 $ where,are the expected values for each are $ 3.00 $. \\
        Also, $ 0.95 $ level of confidence for $\sigma$ is nothing but an interval between $ 0.025 , 0.975 $. 
       \end{block}
\end{frame}




    \begin{block}{}
            The Confidence interval for the variance is given by:
            
               \begin{align}
                   \frac{k}{\chi^{2}_{0.025}} > \sigma^{2} > \frac{k}{\chi^{2}_{0.975}}
                   \label{form_1}
               \end{align}
               
    \end{block}
\end{frame}




    \begin{block}{}
         $\chi^{2}_{0.025}$ and $\chi^{2}_{0.975}$ can be calculated respectively from Fig\ref{fig:my_label1} and Fig\ref{fig:my_label2} for values of v = $4$ and the critical probability from above 
         
         \begin{align}
             \chi^{2}_{0.025} &= 0.484  \\
             \chi^{2}_{0.975} &= 11.143
         \end{align}
         
    \end{block}

\begin{frame}{}
    \begin{block}{}
       the value of k is given by n$\times$v \\
       where v is the variance of the observations and n is the no of observations
       
       \begin{align*}
           k= 4\brak{ (2.90 - 3.00)^{2} + (3.15 - 3.00)^{2} + (3.05 - 3.00)^{2} + (2.96 - 3.00)^{2} }
       \end{align*}
       
       which on calculating we will get 
       
       \begin{align}
           k = 0.0366
       \end{align}
       
    \end{block}
\end{frame}



\begin{frame}

     \begin{figure}
           \centering
           \includegraphics[width= 10cm, height=7cm]{fig1.png}
           \caption{ lower tail critical values of ${\chi}^{2}$ with v degrees of freedom }
          \label{fig:my_ label1}
      \end{figure}
      
\end{frame}



\begin{frame}

     \begin{figure}
           \centering
           \includegraphics[width= 10cm, height=7cm]{fig2.png}
           \caption{{ lower tail critical values of ${\chi}^{2}$ with v degrees of freedom }}
           \label{fig:my_label2}
      \end{figure}
      
\end{frame}



\begin{}{Substituting and solving}
     \begin{block}{Solving}
        On substituting all values in Eq\eqref{form_1} we will get
        
        \begin{align}
            \frac{0.0366}{0.484} > \sigma^{2} > \frac{0.0366}{11.143}
            \label{form_2}
        \end{align}
        
        on simplyfying Eq\eqref{form_2} we will get
        
        \begin{align}
           0.275  > \sigma > 0.057 
        \end{align}
        
        or simply
        
        \begin{align}
            0.057 < \sigma < 0.275
        \end{align}
        
     \end{block}
\end{frame}



\end{document}